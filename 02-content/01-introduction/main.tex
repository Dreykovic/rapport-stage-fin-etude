\chapter*{Introduction Générale}
\addcontentsline{toc}{chapter}{Introduction Générale}
\thispagestyle{empty}
\clearpage
\section*{contexte}
\thispagestyle{empty}


Dans le cadre de leur formation, les étudiants en troisième année de l'Institut de Formation aux Normes et Technologies de l'Informatique doivent effectuer un stage de fin d’études au sein d’une organisation avant l’obtention de leur diplôme. Cette immersion professionnelle vise à leur permettre d’acquérir des compétences pratiques et à mieux appréhender le fonctionnement du monde du travail.\\

C’est dans cette perspective que le stage a été réalisé au sein de l’\ac{ARCOP}. Cette institution joue un rôle central dans la gestion et la régulation des marchés publics, veillant à leur transparence et à leur efficacité. En 2023, l'\ac{ARCOP} a marqué une avancée significative avec l’application d’une nouvelle réglementation relative aux marchés publics, la formation de plus de 1600 acteurs du secteur, ainsi que l’initiation de plusieurs projets de digitalisation et d’amélioration des procédures.\\

Le présent rapport a pour objectif de présenter les différentes missions effectuées au sein de l'\ac{ARCOP}, ainsi que les réalisations techniques développées au cours du stage. Il met en lumière les compétences acquises et les défis rencontrés, tout en illustrant l’apport de cette expérience dans le cadre de la formation académique.\\

\section*{Objectifs}

L’objectif principal de ce stage de fin d’études est de permettre à l’étudiant de mettre en pratique ce qu’il aura acquis tout au long de sa formation en licence d’informatique des organisations à l’\ac{IFNTI}, que ce soit en termes de compétences techniques comme de compétences humaines et professionnelles. L’étudiant devra être capable de s’adapter rapidement aux besoins de l’organisation afin de s’intégrer aux projets sur lesquels il travaillera. La prise en compte des démarches qualités et des méthodes associées au sein de l’organisation est primordiale. Enfin, ce stage doit permettre à l’étudiant d’acquérir une certaine autonomie technique.\\
Tout au long du stage, l’étudiant doit être accompagné d’un maitre de stage ayant entre autres pour role de l’accompagner dans son intégration au fonctionnement de l’organisation, mais aussi de le suivre en se rendant disponible.