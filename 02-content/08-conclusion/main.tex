\chapter{Conclusion Générale}
\clearpage
Ce stage de fin d'études au sein de l'Autorité de Régulation de la Commande Publique (ARCOP) a constitué une expérience professionnelle enrichissante, tant sur le plan technique que personnel. Au terme de ces six mois d'immersion, il convient de dresser un bilan des réalisations effectuées et des enseignements tirés.

La mission principale consistait à développer une solution digitale, OptiHR, pour moderniser la gestion des ressources humaines de l'ARCOP. À travers une démarche méthodique allant de l'analyse des besoins jusqu'à la mise en œuvre technique, nous avons conçu une application répondant précisément aux attentes exprimées par le département des RH et la direction générale. Le système développé permet désormais une gestion dématérialisée des congés et absences, des dossiers personnels, ainsi qu'une diffusion efficace des informations administratives.

L'architecture mise en place, s'appuyant sur des technologies modernes comme Laravel et PostgreSQL, garantit la robustesse, la sécurité et l'évolutivité de la solution. Le respect des bonnes pratiques de développement et l'optimisation des performances ont été des préoccupations constantes tout au long du projet, permettant de livrer un produit de qualité professionnelle.

Ce stage a également été l'occasion de découvrir le fonctionnement d'une institution publique et de contribuer à d'autres activités techniques au sein de l'ARCOP, notamment le support aux utilisateurs de la PASSE et la maintenance des équipements informatiques. Ces missions complémentaires ont permis d'acquérir une vision plus large des enjeux informatiques dans un contexte organisationnel.

Sur le plan des compétences, cette expérience m'a permis de consolider mes connaissances techniques en développement web, en conception logicielle et en gestion de bases de données. Elle m'a également permis de développer des aptitudes essentielles comme l'autonomie, la communication professionnelle et la gestion de projet.

Les défis rencontrés, notamment dans l'intégration du système au sein de l'infrastructure existante et dans la prise en compte des spécificités des processus métier, ont constitué de précieuses opportunités d'apprentissage. Les solutions mises en œuvre pour surmonter ces obstacles ont renforcé ma capacité d'adaptation et de résolution de problèmes.

En perspective, le système OptiHR pourrait être enrichi de nouvelles fonctionnalités, comme l'intégration d'un module de gestion des formations ou d'évaluation des performances, offrant ainsi à l'ARCOP une solution complète et intégrée pour la gestion de ses ressources humaines.

Cette première expérience professionnelle significative conforte mon intérêt pour le développement d'applications métier, domaine dans lequel je souhaite poursuivre ma carrière. Elle m'a permis de mesurer l'importance de concevoir des solutions informatiques qui répondent véritablement aux besoins des utilisateurs et qui s'intègrent harmonieusement dans leur environnement de travail.

Je tiens enfin à souligner que ce stage a été rendu possible grâce à la qualité de l'encadrement et à la confiance qui m'a été accordée par l'équipe de l'ARCOP. Cette collaboration fructueuse témoigne de l'importance des partenariats entre les institutions de formation et les organisations professionnelles pour préparer efficacement les étudiants à leur future carrière.
\clearpage
