\section{Problèmes rencontrés et solutions adoptées}

\subsection{Difficultés techniques}

\subsubsection{Gestion des relations hiérarchiques complexes}
La modélisation des relations hiérarchiques pour le workflow de validation des congés s'est avérée complexe en raison de la structure organisationnelle de l'ARCOP. Ce défi a été résolu par :

\begin{itemize}
    \item L'implémentation d'un modèle de données flexible permettant de représenter les relations N+1
    \item Le développement d'un algorithme recursif pour déterminer le chemin de validation approprié
    \item La création d'un système de règles configurable pour gérer les cas particuliers
\end{itemize}

\subsubsection{Performance des requêtes de filtrage des recours}
Les fonctionnalités de filtrage avancé des recours ont initialement entraîné des problèmes de performance lors du traitement de grands volumes de données. Ces problèmes ont été résolus par :

\begin{itemize}
    \item L'optimisation des requêtes SQL avec des index appropriés
    \item L'implémentation d'un système de mise en cache des résultats fréquemment consultés
    \item Le chargement progressif (lazy loading) des données dans l'interface utilisateur
    \item La pagination efficace des résultats de recherche
\end{itemize}

\subsection{Optimisation et amélioration du code}

\subsubsection{Refactorisation des contrôleurs}
Au cours du développement, nous avons constaté que certains contrôleurs devenaient trop volumineux et difficiles à maintenir. Une refactorisation a été entreprise pour :

\begin{itemize}
    \item Appliquer le principe de responsabilité unique (SRP)
    \item Extraire la logique métier vers des classes de services dédiées
    \item Standardiser la gestion des erreurs et des réponses
    \item Améliorer la testabilité du code
\end{itemize}

\subsubsection{Mise en place d'un système de cache}
Pour améliorer les performances globales de l'application, nous avons implémenté une stratégie de mise en cache :

\begin{itemize}
    \item Cache de requêtes pour les données statiques ou rarement modifiées
    \item Cache de vues pour les éléments d'interface fréquemment utilisés
    \item Mise en cache des résultats de calculs coûteux (statistiques, tableaux de bord)
    \item Implémentation de l'invalidation intelligente du cache lors des modifications
\end{itemize}
