\subsection{Présentation de l'architecture globale}

Le système OptiHR adopte une architecture MVC (Modèle-Vue-Contrôleur)
implémentée avec le framework Laravel, permettant une séparation claire des
responsabilités entre les différentes couches de l'application.

\subsubsection{Architecture MVC}
L'architecture MVC s'articule autour de trois composants principaux :

\begin{itemize}
    \item \textbf{Modèle} : Représente les données et la logique métier de l'application. Dans notre implémentation, les modèles sont matérialisés par des classes Eloquent qui interagissent avec la base de données PostgreSQL.

    \item \textbf{Vue} : Gère l'interface utilisateur et la présentation des données. Nous utilisons le moteur de template Blade, couplé à Bootstrap pour créer une interface responsive et ergonomique.

    \item \textbf{Contrôleur} : Agit comme intermédiaire entre le Modèle et la Vue, traitant les requêtes HTTP, manipulant les données via les modèles et retournant les vues appropriées.
\end{itemize}

\subsubsection{Structure de l'application}
L'application est organisée selon les standards Laravel, avec une structure de
répertoires claire :

\begin{tcolorbox}[colback=black!5!white, colframe=black!75!white, title=Structure du projet OptiHR, fonttitle=\bfseries]
    \texttt{
        app/ \\
        |-- Http/ \\
        |   |-- Controllers/  \quad \# Contrôleurs (Absence, Employee, Document, etc.) \\
        |   |-- Middleware/   \quad \# Middlewares d'authentification et sécurité \\
        |   `-- Requests/     \quad \# Classes de validation des formulaires \\
        |-- Models/           \quad \# Modèles Eloquent (Absence, Department, etc.) \\
        |-- Services/         \quad \# Services spécialisés (AbsencePdfService, etc.) \\
        |-- Console/          \quad \# Commandes Artisan (UpdateAbsenceBalance, etc.) \\
        |-- Mail/             \quad \# Templates d'emails (AbsenceRequestUpdated, etc.) \\
        |-- Observers/        \quad \# Observateurs de modèles \\
        `-- Providers/        \quad \# Service providers Laravel \\
        \\
        bootstrap/ \\
        |-- app.php           \quad \# Point d'entrée de l'application \\
        `-- cache/            \quad \# Cache de configuration \\
        \\
        config/               \quad \# Fichiers de configuration \\
        \\
        database/ \\
        |-- migrations/       \quad \# Structure des tables \\
        |-- seeders/          \quad \# Données initiales \\
        `-- factories/        \quad \# Factories pour les tests \\
        \\
        lang/                 \quad \# Fichiers de traduction (en, fr) \\
    }
\end{tcolorbox}

\vspace{0.5cm}

\begin{tcolorbox}[colback=black!5!white, colframe=black!75!white, title=Suite de la structure du projet OptiHR, fonttitle=\bfseries]
    \texttt{
        Modules/ \\
        |-- Recours/          \quad \# Module de gestion des recours \\
        |   |-- App/          \quad \# Logique métier du module \\
        |   |-- Database/     \quad \# Migrations et seeders spécifiques \\
        |   |-- resources/    \quad \# Vues et assets du module \\
        |   `-- routes/       \quad \# Routes spécifiques au module \\
        \\
        public/ \\
        |-- app-js/           \quad \# Scripts JavaScript spécifiques \\
        |-- assets/           \quad \# Assets statiques (CSS, bundles, fonts) \\
        |-- index.php         \quad \# Point d'entrée HTTP \\
        `-- storage/          \quad \# Fichiers publics (lien symbolique) \\
        \\
        resources/ \\
        |-- css/              \quad \# Styles CSS source \\
        |-- js/               \quad \# Scripts JavaScript source \\
        |-- lang/             \quad \# Traductions additionnelles \\
        `-- views/            \quad \# Templates Blade \\
        |-- components/   \quad \# Composants réutilisables \\
        |-- emails/       \quad \# Templates d'emails \\
        |-- errors/       \quad \# Pages d'erreur \\
        |-- pages/        \quad \# Pages principales \\
        |-- partials/     \quad \# Fragments réutilisables \\
        `-- pdf/          \quad \# Templates pour génération PDF \\
    }
\end{tcolorbox}

\vspace{0.5cm}

\begin{tcolorbox}[colback=black!5!white, colframe=black!75!white, title=Fin de la structure du projet OptiHR, fonttitle=\bfseries]
    \texttt{
        routes/ \\
        |-- web.php           \quad \# Routes web principales \\
        |-- api.php           \quad \# Routes API \\
        `-- channels.php      \quad \# Routes de diffusion WebSockets \\
        \\
        storage/ \\
        |-- app/              \quad \# Fichiers uploadés \\
        |-- framework/        \quad \# Fichiers générés par Laravel \\
        `-- logs/             \quad \# Journaux d'événements \\
        \\
        tests/                \quad \# Tests automatisés \\
        \\
        vendor/               \quad \# Dépendances (Composer) \\
    }
\end{tcolorbox}

\subsubsection{Flux de données}
Le flux de données dans l'application suit un parcours standard dans
l'architecture MVC :

\begin{enumerate}
    \item L'utilisateur envoie une requête HTTP à l'application via son navigateur.
    \item La requête est acheminée vers le contrôleur approprié selon la définition des
          routes.
    \item Le contrôleur traite la requête, fait appel aux services et modèles
          nécessaires.
    \item Les modèles interagissent avec la base de données PostgreSQL via Eloquent ORM.
    \item Le contrôleur prépare les données et les transmet à la vue.
    \item La vue génère le HTML qui est renvoyé au navigateur de l'utilisateur.
\end{enumerate}

Cette architecture modulaire facilite la maintenance, les tests et
l'évolutivité du système, tout en permettant une séparation claire des
responsabilités.
\clearpage