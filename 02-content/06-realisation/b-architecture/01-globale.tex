
\subsection{Présentation de l'architecture globale}
Le portail d'applications développé repose sur une architecture monolithique modulaire basée sur le framework Laravel. Cette approche présente plusieurs avantages :

\begin{itemize}
    \item \textbf{Cohérence technique} : L'ensemble du système utilise les mêmes technologies et patterns de développement.
    \item \textbf{Simplicité de déploiement} : Une seule application à déployer et maintenir.
    \item \textbf{Partage efficace des composants} : Les services communs sont facilement réutilisables entre les modules.
    \item \textbf{Performance optimisée} : Pas de surcoût lié aux communications inter-services.
\end{itemize}

L'architecture suit le pattern MVC (Modèle-Vue-Contrôleur) proposé par Laravel, avec une organisation modulaire permettant d'isoler les fonctionnalités spécifiques à OptiHR et à la Gestion des Recours, tout en partageant les composants communs.

\subsection{Organisation du code source}
Le code source du projet a été organisé selon la structure suivante :







\begin{tcolorbox}[title=Structure du projet Laravel,
    breakable=true,  % Allow the box to break across pages
    width=\textwidth, % Full width
    before skip=10pt,
    after skip=10pt,
    colback=white,
    colframe=black!75]
\begin{lstlisting}[basicstyle=\small\ttfamily, % Smaller font size
     breaklines=true,   % Break long lines
     postbreak=\mbox{\textcolor{red}{$\hookrightarrow$}\space}, % Show break symbol
     columns=flexible,   % Better spacing
     keepspaces=true]    % Preserve spaces
.
+-- app
|   +-- Config
|   |   +-- ActivityLogActions.php
|   +-- Console
|   |   +-- Commands
|   |   |   +-- CleanupActivityLogs.php
|   |   |   +-- SendDailyAppealReminderEmail.php
|   |   |   +-- UpdateAbsenceBalance.php
|   |   |   +-- UpdateAppealDayCount.php
|   |   +-- Kernel.php
|   +-- Exceptions
|   |   +-- Handler.php
|   +-- Http
|   |   +-- Controllers
|   |   |   +-- ActivityLogController.php
|   |   |   +-- AuthController.php
|   |   |   +-- Controller.php
|   |   |   +-- HomeController.php
|   |   |   +-- OptiHR
|   |   |   |   +-- AbsenceController.php
|   |   |   |   +-- AbsenceTypeController.php
|   |   |   |   +-- ...
|   |   |   +-- Recours
|   |   |   |   +-- DacController.php
|   |   |   |   +-- RecoursController.php
|   |   |   |   +-- StatsController.php
|   |   |   +-- RoleController.php
|   |   |   +-- UserController.php
|   |   +-- Kernel.php
|   |   +-- Middleware
|   |   +-- Requests
|   |       +-- StoreAccountRequest.php
|   +-- Jobs
|   |   +-- CleanupActivityLogsJob.php
|   +-- Mail
|   |   +-- AbsenceRequestCreated.php
|   +-- Models
|   |   +-- ActivityLog.php
|   |   +-- OptiHR
|   |   |   +-- Absence.php
|   |   |   +-- AbsenceType.php
|   |   |   +-- ...
|   |   +-- Recours
|   |   |   +-- Appeal.php
|   |   |   +-- ...
|   |   +-- User.php
|   +-- Observers
|   |   +-- EmployeeObserver.php
|   +-- Providers
|   +-- Services
|   +-- Traits
|   |   +-- LogsActivity.php
|   +-- Utils
|       +-- helpers.php
+-- architecture.txt
+-- artisan
+-- bootstrap/
+-- composer.json
+-- composer.lock
+-- database/
+-- lang/
+-- package.json
+-- package-lock.json
+-- palette.scss
+-- phpunit.xml
+-- README.md
+-- resources
|   +-- css/
|   +-- js/
|   +-- lang/
|   +-- views
|       +-- auth
|       +-- base.blade.php
|       +-- components
|       +-- errors
|       +-- modules
|       |   +-- gateway
|       |   +-- opti-hr
|       |   |   +-- emails
|       |   |   +-- pages
|       |   |   +-- partials
|       |   |   +-- pdf
+-- routes
+-- tests
+-- vite.config.js
+-- vite-module-loader.js
\end{lstlisting}
\end{tcolorbox}

Cette organisation du code source a permis de maintenir une séparation claire des responsabilités tout en facilitant la maintenance et l'évolution du système.
