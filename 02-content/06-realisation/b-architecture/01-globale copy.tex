\subsection{Présentation de l'architecture globale}

Le système OptiHR adopte une architecture MVC (Modèle-Vue-Contrôleur) implémentée avec le framework Laravel, permettant une séparation claire des responsabilités entre les différentes couches de l'application.

\subsubsection{Architecture MVC}
L'architecture MVC s'articule autour de trois composants principaux :

\begin{itemize}
    \item \textbf{Modèle} : Représente les données et la logique métier de l'application. Dans notre implémentation, les modèles sont matérialisés par des classes Eloquent qui interagissent avec la base de données PostgreSQL.
    
    \item \textbf{Vue} : Gère l'interface utilisateur et la présentation des données. Nous utilisons le moteur de template Blade, couplé à Bootstrap pour créer une interface responsive et ergonomique.
    
    \item \textbf{Contrôleur} : Agit comme intermédiaire entre le Modèle et la Vue, traitant les requêtes HTTP, manipulant les données via les modèles et retournant les vues appropriées.
\end{itemize}

\subsubsection{Structure de l'application}
L'application est organisée selon les standards Laravel, avec une structure de répertoires claire :

\begin{tcolorbox}[colback=black!5!white, colframe=black!75!white, title=Structure du projet OptiHR, fonttitle=\bfseries]
\texttt{
app/ \\
|-- Http/ \\
|   |-- Controllers/  \quad \# Contrôleurs de l'application \\
|   |-- Middleware/   \quad \# Middlewares personnalisés \\
|   `-- Requests/     \quad \# Classes de validation des requêtes \\
|-- Models/           \quad \# Modèles Eloquent \\
|-- Services/         \quad \# Services métier \\
|-- Repositories/     \quad \# Couche d'accès aux données \\
|-- Providers/        \quad \# Service providers Laravel \\
`-- Events/           \quad \# Événements et listeners \\
\\
database/ \\
|-- migrations/       \quad \# Migrations de base de données \\
|-- seeders/          \quad \# Seeders pour les données de test \\
`-- factories/        \quad \# Factories pour les tests \\
\\
resources/ \\
|-- views/            \quad \# Templates Blade \\
|-- js/               \quad \# Scripts JavaScript \\
`-- css/              \quad \# Styles CSS \\
\\
routes/ \\
|-- web.php           \quad \# Routes web de l'application \\
`-- api.php           \quad \# Routes API (utilisées pour AJAX) \\
\\
tests/                \quad \# Tests automatisés \\
}
\end{tcolorbox}

\subsubsection{Flux de données}
Le flux de données dans l'application suit un parcours standard dans l'architecture MVC :

\begin{enumerate}
    \item L'utilisateur envoie une requête HTTP à l'application via son navigateur.
    \item La requête est acheminée vers le contrôleur approprié selon la définition des routes.
    \item Le contrôleur traite la requête, fait appel aux services et modèles nécessaires.
    \item Les modèles interagissent avec la base de données PostgreSQL via Eloquent ORM.
    \item Le contrôleur prépare les données et les transmet à la vue.
    \item La vue génère le HTML qui est renvoyé au navigateur de l'utilisateur.
\end{enumerate}

Cette architecture modulaire facilite la maintenance, les tests et l'évolutivité du système, tout en permettant une séparation claire des responsabilités.