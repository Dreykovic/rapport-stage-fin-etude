% \subsection{Développement et implémentation}

% Après la mise en place de l'environnement de développement, nous avons procédé à l'implémentation des fonctionnalités clés du système. Cette section détaille les principales fonctionnalités développées pour les modules OptiHR et Gestion des Recours.

% \subsubsection{Portail d'accueil et authentification}

% \paragraph{Interface de connexion}
% La page de connexion constitue le point d'entrée unique du système. Elle a été conçue pour être à la fois esthétique et fonctionnelle.

% \begin{figure}[ht]
%     \centering
%     \includegraphics[width=0.8\textwidth]{images/interface_connexion.png}
%     \caption{Interface de connexion du portail}
%     \label{fig:interface_connexion}
% \end{figure}

% \textbf{Caractéristiques techniques implémentées:}
% \begin{itemize}
%     \item Authentification sécurisée avec Laravel Sanctum
%     \item Protection contre les attaques par force brute (limitation des tentatives)
%     \item Récupération de mot de passe par e-mail
%     \item Mémorisation de session avec token sécurisé
% \end{itemize}

% \paragraph{Tableau de bord principal}
% Après authentification, l'utilisateur accède au tableau de bord central qui présente une vue synthétique du système et donne accès aux différents modules.

% \begin{figure}[ht]
%     \centering
%     \includegraphics[width=0.8\textwidth]{images/tableau_bord_principal.png}
%     \caption{Tableau de bord principal du portail}
%     \label{fig:tableau_bord_principal}
% \end{figure}

% \textbf{Éléments implémentés:}
% \begin{itemize}
%     \item Widgets personnalisés selon le profil utilisateur
%     \item Accès rapide aux modules disponibles
%     \item Notifications en temps réel
%     \item Statistiques globales pour les administrateurs
% \end{itemize}

% \subsubsection{Module OptiHR}

% \paragraph{Gestion des congés et absences}

% Cette fonctionnalité constitue le cœur du module OptiHR et permet la dématérialisation complète du processus de demande et validation des congés.

% \subparagraph{Formulaire de demande de congés}
% Le formulaire de demande a été conçu pour être intuitif tout en capturant l'ensemble des informations nécessaires.

% \begin{figure}[ht]
%     \centering
%     \includegraphics[width=0.8\textwidth]{images/formulaire_conges.png}
%     \caption{Formulaire de demande de congés}
%     \label{fig:formulaire_conges}
% \end{figure}

% \textbf{Fonctionnalités implémentées:}
% \begin{itemize}
%     \item Sélection du type de congé avec règles spécifiques par type
%     \item Calcul automatique de la durée (jours ouvrés)
%     \item Vérification instantanée du solde disponible
%     \item Ajout de pièces justificatives
%     \item Prévisualisation avant soumission
% \end{itemize}

% \textbf{Extrait de code pour le calcul des jours ouvrés:}
% \begin{tcolorbox}[colback=black!5!white, colframe=black!75!white, title=Calcul des jours ouvrés, fonttitle=\bfseries]
% \begin{verbatim}
% /**
%  * Calcule le nombre de jours ouvrés entre deux dates
%  * @param DateTime $startDate Date de début
%  * @param DateTime $endDate Date de fin
%  * @return int Nombre de jours ouvrés
%  */
% private function calculateWorkingDays($startDate, $endDate) {
%     $workingDays = 0;
%     $currentDate = clone $startDate;
    
%     while ($currentDate <= $endDate) {
%         // 6 = samedi, 7 = dimanche
%         $dayOfWeek = $currentDate->format('N');
        
%         if ($dayOfWeek < 6 && !$this->isHoliday($currentDate)) {
%             $workingDays++;
%         }
        
%         $currentDate->modify('+1 day');
%     }
    
%     return $workingDays;
% }
% \end{verbatim}
% \end{tcolorbox}

% \subparagraph{Circuit de validation des demandes}
% Le workflow de validation implémenté respecte strictement la hiérarchie organisationnelle de l'ARCOP.

% \begin{figure}[ht]
%     \centering
%     \includegraphics[width=0.8\textwidth]{images/circuit_validation.png}
%     \caption{Interface de validation des demandes de congés}
%     \label{fig:circuit_validation}
% \end{figure}

% \textbf{Processus implémenté:}
% \begin{enumerate}
%     \item Soumission par l'employé (statut: PENDING)
%     \item Validation par le supérieur hiérarchique N+1 (statut: MANAGER\_APPROVED)
%     \item Approbation par le service RH (statut: HR\_APPROVED)
%     \item Validation finale par le Directeur Général (statut: APPROVED)
% \end{enumerate}

% \textbf{Optimisations techniques:}
% \begin{itemize}
%     \item Notifications automatiques à chaque changement d'état
%     \item Suivi en temps réel de l'avancement de la demande
%     \item Commentaires possibles à chaque étape
%     \item Journalisation complète des actions pour audit
% \end{itemize}

% \subparagraph{Tableau de suivi des demandes}
% Pour permettre un suivi efficace, un tableau de bord dédié a été développé.

% \begin{figure}[ht]
%     \centering
%     \includegraphics[width=0.8\textwidth]{images/tableau_suivi_demandes.png}
%     \caption{Tableau de suivi des demandes de congés}
%     \label{fig:tableau_suivi_demandes}
% \end{figure}

% \textbf{Fonctionnalités implémentées:}
% \begin{itemize}
%     \item Filtrage multicritères (statut, période, type)
%     \item Tri personnalisable par colonnes
%     \item Exportation des données (Excel, PDF)
%     \item Indicateurs visuels d'état (code couleur)
%     \item Actions contextuelles selon le rôle de l'utilisateur
% \end{itemize}

% \paragraph{Gestion des documents administratifs}

% Cette fonctionnalité permet la dématérialisation et l'accès sécurisé aux documents RH.

% \subparagraph{Espace documents personnels}
% Chaque employé dispose d'un espace pour consulter ses documents personnels.

% \begin{figure}[ht]
%     \centering
%     \includegraphics[width=0.8\textwidth]{images/espace_documents_personnels.png}
%     \caption{Interface de consultation des documents personnels}
%     \label{fig:espace_documents_personnels}
% \end{figure}

% \textbf{Caractéristiques techniques:}
% \begin{itemize}
%     \item Classement automatique par catégories et dates
%     \item Prévisualisation intégrée des PDF sans téléchargement
%     \item Recherche textuelle dans les métadonnées
%     \item Historique des consultations
% \end{itemize}

% \subparagraph{Interface d'administration documentaire}
% L'interface d'administration permet au service RH de gérer l'ensemble des documents.

% \begin{figure}[ht]
%     \centering
%     \includegraphics[width=0.8\textwidth]{images/administration_documentaire.png}
%     \caption{Interface d'administration documentaire}
%     \label{fig:administration_documentaire}
% \end{figure}

% \textbf{Fonctionnalités implémentées:}
% \begin{itemize}
%     \item Téléversement par lot avec drag \& drop
%     \item Reconnaissance automatique des employés via convention de nommage
%     \item Catégorisation automatique ou manuelle
%     \item Gestion des versions de documents
%     \item Contrôles d'accès granulaires
% \end{itemize}

% \textbf{Intégration avec Sage Paie:}\\
% Un processus semi-automatisé a été développé pour faciliter l'importation des bulletins:
% \begin{enumerate}
%     \item Export batch depuis Sage au format PDF
%     \item Téléversement groupé via interface glisser-déposer
%     \item Reconnaissance automatique des employés via convention de nommage
%     \item Notification aux employés de la disponibilité des nouveaux bulletins
% \end{enumerate}

% \textbf{Sécurisation des données sensibles:}
% \begin{itemize}
%     \item Chiffrement AES-256 des fichiers stockés
%     \item Journalisation complète des accès
%     \item Vérification d'intégrité des fichiers (hachage SHA-256)
% \end{itemize}

% \subsubsection{Module Gestion des Recours}

% \paragraph{Enregistrement et suivi des recours}

% Cette fonctionnalité permet la gestion complète du cycle de vie des recours déposés auprès de l'ARCOP.

% \subparagraph{Formulaire d'enregistrement des recours}
% Un formulaire complet a été développé pour la saisie précise des informations relatives aux recours.

% \begin{figure}[ht]
%     \centering
%     \includegraphics[width=0.8\textwidth]{images/formulaire_enregistrement_recours.png}
%     \caption{Formulaire d'enregistrement des recours}
%     \label{fig:formulaire_enregistrement_recours}
% \end{figure}

% \textbf{Caractéristiques techniques:}
% \begin{itemize}
%     \item Système de numérotation automatique (format: REC-YYYY-NNNN)
%     \item Validation rigoureuse côté client et serveur
%     \item Champs intelligents avec suggestions (autorités contractantes, types de recours)
%     \item Téléversement sécurisé des pièces jointes avec vérification de type
% \end{itemize}

% \textbf{Extrait de code pour la génération du numéro de recours:}
% \begin{tcolorbox}[colback=black!5!white, colframe=black!75!white, title=Génération du numéro de recours, fonttitle=\bfseries]
% \begin{verbatim}
% /**
%  * Génère un numéro unique de recours
%  * @return string Numéro au format REC-YYYY-NNNN
%  */
% public function generateAppealNumber()
% {
%     $year = date('Y');
%     $latestAppeal = Appeal::whereYear('created_at', $year)
%                           ->orderBy('id', 'desc')
%                           ->first();
    
%     $sequence = 1;
%     if ($latestAppeal) {
%         // Extract sequence from last number (REC-YYYY-NNNN)
%         $parts = explode('-', $latestAppeal->appeal_number);
%         if (count($parts) === 3) {
%             $sequence = intval($parts[2]) + 1;
%         }
%     }
    
%     // Format with leading zeros (0001, 0002, etc.)
%     return 'REC-' . $year . '-' . str_pad($sequence, 4, '0', STR_PAD_LEFT);
% }
% \end{verbatim}
% \end{tcolorbox}

% \subparagraph{Interface de suivi des recours}
% Une interface dédiée permet le suivi précis de l'état d'avancement des recours.

% \begin{figure}[ht]
%     \centering
%     \includegraphics[width=0.8\textwidth]{images/suivi_recours.png}
%     \caption{Interface de suivi des recours}
%     \label{fig:suivi_recours}
% \end{figure}

% \textbf{Fonctionnalités implémentées:}
% \begin{itemize}
%     \item Vue détaillée de chaque recours avec historique des actions
%     \item Système d'alertes sur les délais (visuel et notifications)
%     \item Journal chronologique des événements
%     \item Génération automatique des documents officiels (accusés de réception, convocations)
% \end{itemize}

% \textbf{Optimisations techniques:}
% \begin{itemize}
%     \item Mises à jour en temps réel via WebSocket
%     \item Cache intelligent pour les données statiques
%     \item Pagination efficace pour les grands volumes
%     \item Affichage conditionnel selon les permissions
% \end{itemize}

% \paragraph{Tableaux de bord analytiques}

% Cette fonctionnalité offre une vision stratégique de l'activité des recours à travers des visualisations avancées.

% \subparagraph{Vue d'ensemble statistique}
% L'écran principal présente une synthèse visuelle de l'activité.

% \begin{figure}[ht]
%     \centering
%     \includegraphics[width=0.8\textwidth]{images/tableau_bord_statistique.png}
%     \caption{Tableau de bord statistique des recours}
%     \label{fig:tableau_bord_statistique}
% \end{figure}

% \textbf{Visualisations implémentées:}
% \begin{itemize}
%     \item Répartition des recours par statut (diagramme circulaire)
%     \item Évolution temporelle du nombre de recours (graphique linéaire)
%     \item Délais moyens de traitement par type (diagramme à barres)
%     \item Cartographie des autorités contractantes concernées
% \end{itemize}

% \textbf{Aspects techniques:}
% \begin{itemize}
%     \item Génération dynamique via Chart.js avec thème personnalisé
%     \item Agrégations SQL optimisées côté serveur
%     \item Filtrage AJAX sans rechargement de page
%     \item Exportation des graphiques en format image ou données brutes
% \end{itemize}

% \textbf{Extrait de code pour la génération du graphique de répartition:}
% \begin{tcolorbox}[colback=black!5!white, colframe=black!75!white, title=Génération du graphique de répartition, fonttitle=\bfseries]
% \begin{verbatim}
% function generateStatusDistributionChart(containerId, data) {
%     const ctx = document.getElementById(containerId).getContext('2d');
    
%     // Extraction des données
%     const labels = Object.keys(data);
%     const values = Object.values(data);
    
%     // Configuration des couleurs selon le statut
%     const colors = labels.map(label => {
%         switch(label) {
%             case 'En attente': return '#FFC107';
%             case 'En traitement': return '#2196F3';
%             case 'Approuvé': return '#4CAF50';
%             case 'Rejeté': return '#F44336';
%             default: return '#9E9E9E';
%         }
%     });
    
%     // Création du graphique
%     new Chart(ctx, {
%         type: 'doughnut',
%         data: {
%             labels: labels,
%             datasets: [{
%                 data: values,
%                 backgroundColor: colors,
%                 borderWidth: 1
%             }]
%         },
%         options: {
%             responsive: true,
%             maintainAspectRatio: false,
%             legend: {
%                 position: 'right',
%                 labels: {
%                     padding: 20,
%                     boxWidth: 15
%                 }
%             },
%             tooltips: {
%                 callbacks: {
%                     label: function(tooltipItem, data) {
%                         const dataset = data.datasets[tooltipItem.datasetIndex];
%                         const total = dataset.data.reduce((acc, val) => acc + val, 0);
%                         const currentValue = dataset.data[tooltipItem.index];
%                         const percentage = Math.round((currentValue/total) * 100);
%                         return `${data.labels[tooltipItem.index]}: ${currentValue} (${percentage}%)`;
%                     }
%                 }
%             }
%         }
%     });
% }
% \end{verbatim}
% \end{tcolorbox}

% \subparagraph{Rapports détaillés}
% L'interface permet également la génération de rapports détaillés personnalisables.

% \begin{figure}[ht]
%     \centering
%     \includegraphics[width=0.8\textwidth]{images/rapports_detailles.png}
%     \caption{Interface de génération de rapports détaillés}
%     \label{fig:rapports_detailles}
% \end{figure}

% \textbf{Fonctionnalités implémentées:}
% \begin{itemize}
%     \item Création de rapports sur mesure avec filtres avancés
%     \item Programmation de rapports récurrents (quotidiens, hebdomadaires, mensuels)
%     \item Multiples formats d'export (PDF, Excel, CSV)
%     \item Partage direct par email avec droits d'accès configurables
% \end{itemize}

% \subsubsection{Intégration et points de convergence}

% L'architecture du système a été conçue pour assurer une cohérence globale tout en permettant l'évolution indépendante des modules.

% \paragraph{Système de notifications centralisé}
% Un point crucial d'intégration est le système de notifications qui consolide les alertes provenant des différents modules.

% \begin{figure}[ht]
%     \centering
%     \includegraphics[width=0.8\textwidth]{images/centre_notifications.png}
%     \caption{Centre de notifications unifié}
%     \label{fig:centre_notifications}
% \end{figure}

% \textbf{Caractéristiques techniques:}
% \begin{itemize}
%     \item Notifications en temps réel via WebSocket
%     \item Multiples canaux (in-app, email, exportable)
%     \item Préférences configurables par utilisateur
%     \item Agrégation intelligente pour éviter la surcharge d'informations
% \end{itemize}

% \paragraph{Interfaces utilisateur cohérentes}
% Une attention particulière a été portée à l'homogénéité des interfaces utilisateur.

% \textbf{Composants partagés développés:}
% \begin{itemize}
%     \item En-têtes et pieds de page standardisés
%     \item Formulaires avec validation harmonisée
%     \item Tableaux de données avec fonctionnalités communes
%     \item Modales et dialogues de confirmation
%     \item Indicateurs visuels d'état et de progression
% \end{itemize}

% \textbf{Optimisations de performance frontend:}
% \begin{itemize}
%     \item Minification et bundling des assets via Laravel Mix
%     \item Chargement conditionnel des scripts JS
%     \item Lazy loading des images et composants lourds
%     \item Mise en cache des templates Blade compilés
% \end{itemize}

% \subsubsection{Optimisations techniques globales}

% Tout au long du développement, plusieurs optimisations ont été implémentées pour garantir performance et sécurité.

% \paragraph{Optimisations de base de données}
% \begin{itemize}
%     \item Indexation stratégique des champs fréquemment recherchés
%     \item Requêtes N+1 éliminées par utilisation d'eager loading
%     \item Transactions pour garantir l'intégrité des données
%     \item Mises à jour par lots pour les opérations massives
% \end{itemize}

% \paragraph{Sécurité renforcée}
% \begin{itemize}
%     \item Validation stricte des entrées à tous les niveaux
%     \item Protection CSRF sur tous les formulaires
%     \item Échappement systématique des sorties pour prévenir les XSS
%     \item Rate limiting sur les endpoints sensibles
%     \item Logs de sécurité pour audit et détection d'intrusion
% \end{itemize}

% \paragraph{Tests automatisés}
% Une suite de tests a été développée pour garantir la fiabilité du système:
% \begin{itemize}
%     \item Tests unitaires pour les composants critiques
%     \item Tests d'intégration pour les workflows complexes
%     \item Tests de charge pour valider les performances sous stress
% \end{itemize}

% Ces optimisations garantissent que le système répond non seulement aux besoins fonctionnels actuels de l'ARCOP, mais qu'il reste également performant et évolutif pour les développements futurs.