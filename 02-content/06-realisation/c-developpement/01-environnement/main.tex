subsection{Mise en place de l'environnement de développement}
Pour assurer un développement fluide et efficace, un environnement de travail stable et bien configuré a été mis en place sous Ubuntu 24. Cette section détaille les étapes suivies pour l'installation et la configuration des outils nécessaires.

\subsubsection{Prérequis}
Les outils suivants ont été utilisés :
\begin{itemize}
    \item Système d'exploitation : \textbf{Ubuntu 24}
    \item Éditeur de code : \textbf{Visual Studio Code}
    \item Serveur Web et PHP : \textbf{Apache, PHP 8.2}
    \item Base de données : \textbf{PostgreSQL}
    \item Gestionnaire de paquets : \textbf{Composer (pour PHP) et npm (pour JavaScript)}
    \item Système de versionnement : \textbf{Git et GitHub}
\end{itemize}

\subsubsection{Installation des outils}

\subsubsection*{Installation de PHP 8.2 et Apache}
Ubuntu 24 ne propose pas PHP 8.2 par défaut. Pour l'installer avec Apache :
\begin{tcolorbox}[colback=black, coltext=white, title=Installation de PHP 8.2 et Apache, fonttitle=\bfseries]
\texttt{sudo apt update \&\& sudo apt upgrade -y} \\
\texttt{sudo apt install apache2 php8.2 libapache2-mod-php8.2 php8.2-cli php8.2-mbstring php8.2-xml php8.2-curl php8.2-pgsql unzip -y}
\end{tcolorbox}

Une fois l'installation terminée, redémarrer Apache :

\begin{tcolorbox}[colback=black, coltext=white, title=Redémarrage du serveur Apache, fonttitle=\bfseries]
\texttt{sudo systemctl restart apache2}
\end{tcolorbox}

Vérifier que PHP est bien installé :

\begin{tcolorbox}[colback=black, coltext=white, title=Vérification de la version de PHP, fonttitle=\bfseries]
\texttt{php -v}
\end{tcolorbox}

\subsubsection*{Installation de Composer}
Composer est indispensable pour gérer les dépendances PHP :

\begin{tcolorbox}[colback=black, coltext=white, title=Installation de Composer, fonttitle=\bfseries]
\texttt{curl -sS https://getcomposer.org/installer | php} \\
\texttt{sudo mv composer.phar /usr/local/bin/composer} \\
\texttt{composer -V}
\end{tcolorbox}

\subsubsection*{Installation et configuration de PostgreSQL}
PostgreSQL est utilisé comme base de données :

\begin{tcolorbox}[colback=black, coltext=white, title=Installation de PostgreSQL, fonttitle=\bfseries]
\texttt{sudo apt install postgresql postgresql-contrib -y}
\end{tcolorbox}

Démarrer PostgreSQL et vérifier son installation :

\begin{tcolorbox}[colback=black, coltext=white, title=Démarrage et vérification, fonttitle=\bfseries]
\texttt{sudo systemctl start postgresql} \\
\texttt{sudo systemctl enable postgresql} \\
\texttt{psql --version}
\end{tcolorbox}

\subsubsection*{Création du projet Laravel}
Initialiser un nouveau projet Laravel :

\begin{tcolorbox}[colback=black, coltext=white, title=Création du projet Laravel, fonttitle=\bfseries]
\texttt{composer create-project --prefer-dist laravel/laravel:\^10.0 arcop-portal}
\end{tcolorbox}

Générer la clé d'application :

\begin{tcolorbox}[colback=black, coltext=white, title=Génération de la clé d'application, fonttitle=\bfseries]
\texttt{cd arcop-portal} \\
\texttt{php artisan key:generate}
\end{tcolorbox}

\subsubsection*{Configuration de la base de données}
Configurer la connexion à PostgreSQL dans le fichier .env :

\begin{tcolorbox}[colback=black, coltext=white, title=Configuration de la base de données, fonttitle=\bfseries]
\texttt{DB\_CONNECTION=pgsql} \\
\texttt{DB\_HOST=127.0.0.1} \\
\texttt{DB\_PORT=5432} \\
\texttt{DB\_DATABASE=arcop\_portal} \\
\texttt{DB\_USERNAME=postgres} \\
\texttt{DB\_PASSWORD=votre\_mot\_de\_passe}
\end{tcolorbox}

\subsubsection*{Lancement du serveur de développement}
Démarrer Laravel en mode développement :

\begin{tcolorbox}[colback=black, coltext=white, title=Lancement du serveur Laravel, fonttitle=\bfseries]
\texttt{php artisan serve}
\end{tcolorbox}

L'application est maintenant accessible à l'adresse :

\begin{tcolorbox}[colback=black, coltext=white, title=URL d'accès, fonttitle=\bfseries]
\texttt{http://127.0.0.1:8000/}
\end{tcolorbox}
