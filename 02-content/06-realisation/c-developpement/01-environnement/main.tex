\subsection{Mise en place de l'environnement de développement}

Pour assurer un développement efficace du projet OptiHR, j'ai configuré un environnement de développement complet et cohérent. Cette configuration a permis de garantir un flux de travail fluide et d'éviter les problèmes d'incompatibilité tout au long du cycle de développement.

\subsubsection{Prérequis système}
L'environnement de développement a été configuré avec les outils suivants :
\begin{itemize}
    \item Système d'exploitation : \textbf{Ubuntu 22.04 LTS}
    \item Éditeur de code : \textbf{Visual Studio Code} avec extensions pour PHP, Laravel et Git
    \item Serveur Web : \textbf{Apache 2.4}
    \item Langage et framework : \textbf{PHP 8.1} et \textbf{Laravel 10}
    \item Base de données : \textbf{PostgreSQL 14}
    \item Gestion des dépendances : \textbf{Composer 2.5} et \textbf{npm 8.19}
    \item Contrôle de version : \textbf{Git 2.34}
\end{itemize}

\subsubsection{Installation des composants essentiels}

\paragraph{Configuration du serveur et PHP}
L'installation de PHP et Apache a été réalisée selon les recommandations officielles pour garantir une compatibilité optimale avec Laravel :

\begin{tcolorbox}[colback=black, coltext=white, title=Installation de PHP et Apache, fonttitle=\bfseries]
\texttt{sudo apt update \&\& sudo apt upgrade -y} \\
\texttt{sudo apt install apache2 php8.1 libapache2-mod-php8.1 php8.1-cli} \\
\texttt{sudo apt install php8.1-mbstring php8.1-xml php8.1-curl php8.1-pgsql} \\
\texttt{sudo apt install php8.1-gd php8.1-zip unzip -y}
\end{tcolorbox}

L'activation des modules PHP nécessaires a ensuite été effectuée :

\begin{tcolorbox}[colback=black, coltext=white, title=Activation des modules PHP, fonttitle=\bfseries]
\texttt{sudo phpenmod mbstring xml curl pgsql gd} \\
\texttt{sudo systemctl restart apache2}
\end{tcolorbox}

\paragraph{Installation et configuration de PostgreSQL}
PostgreSQL a été choisi pour sa robustesse et sa conformité ACID, particulièrement importante pour une application de gestion RH :

\begin{tcolorbox}[colback=black, coltext=white, title=Installation de PostgreSQL, fonttitle=\bfseries]
\texttt{sudo apt install postgresql postgresql-contrib -y} \\
\texttt{sudo systemctl start postgresql} \\
\texttt{sudo systemctl enable postgresql}
\end{tcolorbox}

Création de la base de données pour le projet :

\begin{tcolorbox}[colback=black, coltext=white, title=Configuration de la base de données, fonttitle=\bfseries]
\texttt{sudo -i -u postgres} \\
\texttt{createuser --interactive --pwprompt} \\
\texttt{createdb optirh -O optirh\_user} \\
\texttt{exit}
\end{tcolorbox}

\paragraph{Installation de Composer}
Composer a été installé pour gérer efficacement les dépendances PHP du projet :

\begin{tcolorbox}[colback=black, coltext=white, title=Installation de Composer, fonttitle=\bfseries]
\texttt{curl -sS https://getcomposer.org/installer | php} \\
\texttt{sudo mv composer.phar /usr/local/bin/composer} \\
\texttt{composer global require laravel/installer}
\end{tcolorbox}

\subsubsection{Configuration du projet OptiHR}

\paragraph{Clonage et configuration initiale}
Pour initialiser le projet, j'ai cloné le dépôt GitHub et configuré l'environnement :

\begin{tcolorbox}[colback=black, coltext=white, title=Clonage du projet, fonttitle=\bfseries]
\texttt{git clone https://github.com/Dreykovic/optirh.git} \\
\texttt{cd optirh}
\end{tcolorbox}

Installation des dépendances :

\begin{tcolorbox}[colback=black, coltext=white, title=Installation des dépendances, fonttitle=\bfseries]
\texttt{composer install} \\
\texttt{npm install} \\
\texttt{npm run dev}
\end{tcolorbox}

Configuration de l'environnement Laravel :

\begin{tcolorbox}[colback=black, coltext=white, title=Configuration de l'environnement, fonttitle=\bfseries]
\texttt{cp .env.example .env} \\
\texttt{php artisan key:generate}
\end{tcolorbox}

\paragraph{Configuration de la base de données}
J'ai ensuite configuré le fichier .env pour la connexion à PostgreSQL :

\begin{tcolorbox}[colback=black, coltext=white, title=Configuration de la connexion DB, fonttitle=\bfseries]
\texttt{DB\_CONNECTION=pgsql} \\
\texttt{DB\_HOST=127.0.0.1} \\
\texttt{DB\_PORT=5432} \\
\texttt{DB\_DATABASE=optirh} \\
\texttt{DB\_USERNAME=optirh\_user} \\
\texttt{DB\_PASSWORD=*****}
\end{tcolorbox}

Exécution des migrations pour créer les tables nécessaires :

\begin{tcolorbox}[colback=black, coltext=white, title=Exécution des migrations, fonttitle=\bfseries]
\texttt{php artisan migrate} \\
\texttt{php artisan db:seed}
\end{tcolorbox}

\paragraph{Configuration du serveur de développement}
Pour le développement local, j'ai utilisé le serveur intégré de Laravel :

\begin{tcolorbox}[colback=black, coltext=white, title=Lancement du serveur de développement, fonttitle=\bfseries]
\texttt{php artisan serve --host=0.0.0.0 --port=8000}
\end{tcolorbox}

Le serveur de développement est alors accessible à l'adresse \texttt{http://localhost:8000}

\subsubsection{Bonnes pratiques de développement}
Pour garantir la qualité du code, nous avons mis en place plusieurs pratiques collaboratives :

\begin{itemize}
    \item Travail en équipe de deux développeurs avec une répartition claire des tâches
    \item Stratégie Git structurée avec trois niveaux :
    \begin{itemize}
        \item Branche \texttt{main} : version stable et production-ready du code
        \item Branche \texttt{dev} : intégration et tests des nouvelles fonctionnalités
        \item Branches \texttt{feature/*} : développement isolé de chaque fonctionnalité
    \end{itemize}
    \item Utilisation exclusive des pull requests pour l'intégration du code (\texttt{feature/*} → \texttt{dev} → \texttt{main})
    \item Revue de code systématique avant chaque fusion
    \item Standardisation du code avec PHP-CS-Fixer et configuration dans le projet
    \item Documentation des méthodes et classes avec PHPDoc
\end{itemize}

Cette méthodologie rigoureuse a permis d'assurer un développement collaboratif efficace, d'éviter les conflits de code et de maintenir une qualité constante tout au long du projet OptiHR.
\clearpage