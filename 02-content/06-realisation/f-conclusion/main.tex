\section{Conclusion}

La phase de réalisation du portail d'applications intégrant les modules OptiHR et Gestion des Recours a permis de transformer une vision conceptuelle en une solution opérationnelle, répondant précisément aux exigences spécifiques de l'ARCOP.

\subsection{Retour sur le processus technique}

Le cycle de développement a suivi une trajectoire itérative, nous permettant d'ajuster continuellement l'implémentation aux réalités du terrain :

\begin{itemize}
    \item La configuration initiale de l'environnement sur des serveurs Linux a constitué une base technique solide
    \item L'implémentation progressive des fonctionnalités a facilité les retours d'expérience précoces
    \item L'approche modulaire a permis de développer en parallèle les deux modules tout en garantissant leur cohérence
    \item Les défis techniques rencontrés ont stimulé la recherche de solutions innovantes, notamment pour l'intégration des différents workflows
    \item Les phases de test ont validé la robustesse de l'architecture technique choisie
\end{itemize}

\subsection{Apports techniques du projet}

Au-delà des fonctionnalités métier, ce projet a introduit plusieurs avancées techniques au sein de l'ARCOP :

\begin{itemize}
    \item Première application utilisant PostgreSQL comme système de gestion de base de données
    \item Mise en place d'une infrastructure de déploiement automatisé simplifiant les futures mises à jour
    \item Implémentation d'un système de logs centralisé facilitant la maintenance
    \item Introduction de mécanismes de sécurité avancés protégeant l'ensemble de l'écosystème informatique
    \item Création d'un portail modulaire facilitant l'intégration future d'autres applications
\end{itemize}

\subsection{Apprentissages et bonnes pratiques}

Ce processus de développement a également été l'occasion d'instituer des bonnes pratiques qui bénéficieront aux futurs projets :

\begin{itemize}
    \item Adoption d'une approche de développement axée sur les tests (TDD)
    \item Mise en place d'un processus de revue de code systématique
    \item Documentation technique approfondie pour faciliter la transmission de connaissances
    \item Standardisation des pratiques de développement à travers les différents modules
    \item Formation continue de l'équipe informatique interne pour assurer l'autonomie à long terme
\end{itemize}

La réalisation de ce portail d'applications constitue non seulement une solution technique répondant à des besoins spécifiques, mais également un véritable tremplin vers une culture de développement moderne au sein de l'ARCOP. Les enseignements tirés de cette phase de développement éclaireront la stratégie d'évolution et d'amélioration continue du système.
