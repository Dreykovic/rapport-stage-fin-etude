\thispagestyle{empty}
\begin{abstract}
    Le présent rapport expose le déroulement et les réalisations de mon stage de fin d'études effectué à l'Autorité de Régulation de la Commande Publique (ARCOP) du 15 août 2024 au 19 février 2025. Ce stage s'inscrit dans le cadre de l'obtention de ma licence en Informatique, filière Génie Informatique, à l'Institut de Formation aux Normes et Technologies de l'Informatique (IFNTI) de Sokodé.
    
    L'objectif principal de ce stage était de développer et de personnaliser une solution digitale pour le service des Ressources Humaines de l'ARCOP, dénommée OptiHR. Cette application vise à centraliser et automatiser la gestion des congés, absences, dossiers personnels et communications internes, en s'appuyant sur une architecture moderne et évolutive.
    
    Sous la supervision de M. GBOLOVI Komi Dodzi, j'ai mené à bien l'analyse des besoins, la conception et le développement de cette solution. L'application OptiHR implémente un workflow complet de validation des demandes d'absence qui respecte la hiérarchie organisationnelle de l'ARCOP. Elle offre également des fonctionnalités telles que la gestion des documents administratifs, la consultation des dossiers personnels et la diffusion d'informations RH.
    
    Le développement a été réalisé à l'aide de technologies modernes, notamment le framework Laravel, le système de gestion de base de données PostgreSQL et l'interface utilisateur Bootstrap, garantissant ainsi la robustesse, la sécurité et l'évolutivité de la solution.
    
    Au-delà du projet principal, j'ai également participé à diverses tâches techniques au sein de l'ARCOP, comme le support aux utilisateurs de la Plateforme de l'ARCOP pour des Services Sécurisés et Électroniques (PASSE), l'installation et la maintenance des équipements informatiques, et le débogage du réseau téléphonique IP.
    
    Cette expérience professionnelle m'a permis de consolider mes compétences techniques en développement web et en gestion de projets informatiques, tout en acquérant une meilleure compréhension des enjeux organisationnels et des processus métier dans un contexte institutionnel.
    \end{abstract}
\clearpage