\usepackage[utf8]{inputenc}
\usepackage[T1]{fontenc}
\usepackage[french]{babel} % If you write in French
\usepackage{xcolor,graphicx}


\usepackage{geometry}
\geometry{left=2cm, right=2cm, top=4cm, bottom=4cm}
\usepackage{titlesec}
\usepackage{lmodern}
\usepackage{array}
\usepackage{multirow}

\usepackage[hidelinks]{hyperref} % Permet de rendre les liens cliquables sans les entourer d'un cadre coloré


\usepackage{fancyhdr} % Pour personnaliser les en-têtes et pieds de page
% Configuration des en-têtes et pieds de page
\pagestyle{fancy}
\setlength{\headheight}{15.35403pt}
\addtolength{\topmargin}{-3.35403pt}
\fancyhf{} % Efface les en-têtes et pieds de page par défaut
\fancyhead[L]{\leftmark} % Affiche le nom du chapitre à gauche
\fancyfoot[R]{\thepage} % Affiche le numéro de page au centre du pied de page

\usepackage{enumitem}
% Uniquement le premier niveau
\setlist[itemize,1]{label={.}}
% Deuxième niveau avec un point gras
\setlist[itemize,2]{label={\boldmath$\cdot$}}



\usepackage{longtable} % Pour permettre aux tableaux de se répartir sur plusieurs pages


\renewcommand{\arraystretch}{1.5} % Espacement entre les lignes du tableau


\usepackage{booktabs} % Pour un meilleur rendu du tableau
\usepackage{float}
\usepackage{caption}


\usepackage{xcolor}
\usepackage{tcolorbox}
\usepackage{listings}

% Définition d'un style de terminal
\newtcbox{\terminalbox}{on line, colback=black, colframe=black, boxrule=0pt, left=3pt, right=3pt, top=2pt, bottom=2pt, boxsep=3pt, sharp corners, enhanced}


\usepackage{acronym} % Pour gérer les acronymes
