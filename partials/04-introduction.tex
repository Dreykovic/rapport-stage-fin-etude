\chapter{Introduction Générale}
\clearpage
\section{contexte}

Le stage de fin d’études constitue une étape cruciale dans le parcours académique de tout étudiant. Il offre l’opportunité de mettre en pratique les compétences acquises au cours de la formation, tout en découvrant le monde professionnel et ses défis. En fin de troisième année, chaque étudiant doit réaliser un stage d’une durée minimale de six mois dans une entreprise de son choix. Le projet de ce stage consiste principalement à développer une solution informatique répondant à un besoin spécifique de l'entreprise.

Le présent rapport traite du stage effectué au sein de l’entreprise \textbf{ARCOP} (Autorité de Régulation de la Commande Publique), un acteur majeur dans la régulation des marchés publics au Togo. L'objectif principal de ce stage était de participer à la conception et au développement d'une application de gestion des ressources humaines, afin d'automatiser et d'optimiser les processus internes liés à la gestion des employés de l'organisation. Cette application vise à centraliser des tâches telles que la gestion des congés, des absences et des horaires de travail, contribuant ainsi à une gestion plus fluide et plus efficace des ressources humaines.

Dans un contexte où le gouvernement togolais a entrepris une réforme significative de son système de marchés publics pour renforcer la transparence et l'efficacité des procédures, l'ARCOP joue un rôle clé. L'institution veille à la conformité avec les directives de l’Union Économique et Monétaire Ouest Africaine (UEMOA) et assure une régulation efficace du secteur. Mon projet s’inscrit dans cette dynamique, visant à moderniser l’administration interne de l'ARCOP à travers l'implémentation d'une solution informatique pour la gestion des ressources humaines.
\section{Problématique}
\section{Objectifs}
L’objectif principal de ce stage de fin d’études est de permettre à l’étudiant de mettre en pratique ce qu’il aura acquis tout au long de sa formation en licence d’informatique des organisations à l’IFNTI, que ce soit en termes de compétences techniques comme de compétences humaines et professionnelles. L’étudiant devra être capable de s’adapter rapidement aux besoins de l’organisation afin de s’intégrer aux projets sur lesquels il travaillera. La prise en compte des démarches qualités et des méthodes associées au sein de l’organisation est primordiale. Enfin, ce stage doit permettre à l’étudiant d’acquérir une certaine autonomie technique.\\
Tout au long du stage, l’étudiant doit être accompagné d’un maitre de stage ayant entre autres pour role de l’accompagner dans son intégration au fonctionnement de l’organisation, mais aussi de le suivre en se rendant disponible.