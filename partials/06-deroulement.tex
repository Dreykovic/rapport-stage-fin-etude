\chapter{Déroulement factuel du stage}
\clearpage

\section{ Prise de fonction et intégration}
À mon arrivée à l’ARCOP, j’ai été accueilli au sein de la Direction des Statistiques et de la Documentation. Après une présentation de l’organisation et des missions du service, j’ai pris connaissance des objectifs du stage ainsi que des attentes de mon tuteur.

Durant cette phase d’intégration, j’ai également :
\begin{itemize}
    \item Découvert l’environnement de travail et les outils utilisés.
    \item Pris en main les méthodologies internes et les processus en place.
    \item Échangé avec les membres de l’équipe pour mieux comprendre les besoins et le cadre du stage.

\end{itemize}
\section{Découverte du projet principal et études préliminaires}
L’objectif principal de mon stage était la \textbf{digitalisation des processus de travail du département des ressources humaines (RH)}. Avant d’entamer le développement, plusieurs tâches préparatoires ont été réalisées :




\begin{itemize}
    \item Analyse des besoins du département en collaboration avec le responsable RH.
    \item Étude des solutions existantes et identification des axes d’amélioration.
    \item Participation aux réunions pour recueillir les attentes des utilisateurs.
    \item Rédaction d’un cahier des charges et validation des fonctionnalités essentielles avec mon tuteur.
\end{itemize}

\section{Développement de la solution numérique}
Après cette phase d’étude, j’ai entamé la conception et le développement de l’application. Cela a inclus :
\begin{itemize}
    \item La mise en place de l’environnement de développement (choix des technologies, configuration des outils).
    \item La conception de l’architecture du projet et de la base de données.
    \item Le développement des premières fonctionnalités, notamment \textbf{ l’authentification et la gestion des utilisateurs}.
    \item Des tests et des ajustements basés sur les retours du tuteur et des futurs utilisateurs.

\end{itemize}
Par la suite, d’autres fonctionnalités ont été intégrées, comme :

\begin{itemize}
    \item La gestion des documents administratifs.
    \item L’automatisation de certaines tâches RH.
    \item L’amélioration de l’interface utilisateur pour une meilleure expérience.
\end{itemize}


\section{Participation au service de support et aux tâches techniques}
En parallèle du développement de mon projet principal, j’ai apporté mon aide à plusieurs niveaux du support technique, notamment :
\begin{itemize}
    \item \textbf{Support aux utilisateurs de la plateforme PASSE :} 
    \begin{itemize}
        \item Réponse aux demandes des utilisateurs rencontrant des difficultés techniques.
        \item Diagnostic et résolution des bugs signalés.
    \end{itemize}
    \item \textbf{Installation et maintenance des équipements informatiques :} \begin{itemize}
        \item Installation et configuration d’ordinateurs de bureau pour les employés.
        \item Installation et mise à jour d’antivirus pour renforcer la sécurité informatique.
    \end{itemize}
    \item \textbf{Débogage du réseau téléphonique IP de l’entreprise :} \begin{itemize}
        \item Analyse des dysfonctionnements et recherche des causes des interruptions de service.
        \item Participation aux interventions techniques pour rétablir la connexion et améliorer la qualité des appels.
    \end{itemize}
\end{itemize}

Cette expérience m’a offert une meilleure compréhension des défis liés à la gestion d’un service numérique et à l’accompagnement des utilisateurs.
\section{Autres tâches effectuées pendant le stage}
En plus du développement et du support technique, j’ai été impliqué dans plusieurs autres missions :
\begin{itemize}
    \item \textbf{Assistance technique générale} : dépannage et maintenance des outils informatiques du personnel.
  
    \item \textbf{Participation aux réunions }: Présentation de l’avancement du projet et prise en compte des suggestions d’amélioration.
\end{itemize}
\section{Finalisation du projet et restitution}
Dans les dernières semaines du stage, mon travail s’est concentré sur :

\begin{itemize}
    \item La finalisation du développement et la correction des derniers bugs.
    \item La rédaction de la documentation technique et utilisateur.
    \item La présentation du projet aux responsables du département RH et la collecte des feedbacks.
    \item La formation des collaborateurs à l’utilisation de la solution développée.

\end{itemize}
\section{Bilan et conclusion}
À la fin du stage, un bilan a été réalisé avec mon tuteur afin d’évaluer les résultats obtenus. Cette expérience m’a permis de :
\begin{itemize}
    \item Mettre en pratique mes compétences en \textbf{ développement web, maintenance informatique et support technique}.
    \item Mettre en pratique mes compétences en développement web, maintenance informatique et support technique.
    \item Acquérir de nouvelles compétences, notamment en \textbf{ gestion des infrastructures réseau et dépannage technique}.
\end{itemize}

\clearpage
