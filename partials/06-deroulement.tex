\chapter{Déroulement factuel du stage}
\clearpage


\section{Prise de fonction et intégration}
À l'arrivée au sein de l'ARCOP, l'intégration s'est effectuée au sein de la Direction des Statistiques et de la Documentation. Une présentation de l'organisation et des missions du service a permis de mieux comprendre son fonctionnement. Les objectifs du stage ainsi que les attentes du tuteur ont ensuite été exposés.
Durant cette phase d'intégration, plusieurs actions ont été réalisées :

\begin{itemize}
    \item Découvert l'environnement de travail et les outils utilisés.
    \item Pris en main les méthodologies internes et les processus en place.
    \item Échangé avec les membres de l'équipe pour mieux comprendre les besoins et le cadre du stage.

\end{itemize}
\section{Tâches effectuées}
Tout au long du stage à la Direction des Statistiques et de la Documentation de l'ARCOP, plusieurs tâches ont été effectuées, notamment:
\begin{itemize}
    \item Installation et maintenance des équipements informatiques
    \item Développement Web
    \item Débogage du réseau téléphonique IP de l'entreprise 
    \item Support aux utilisateurs de la plateforme PASSE
    \item Assistance technique générale    
\end{itemize} 
\subsection{Installation et maintenance des équipements informatiques}
Cette mission a consisté à assurer le bon fonctionnement du matériel informatique en effectuant :
\begin{itemize}
    \item L’installation et la configuration des ordinateurs de bureau pour les employés.

    \item La mise en place et la mise à jour des logiciels antivirus afin de garantir la sécurité des données.
    \item Le diagnostic et la résolution des pannes matérielles et logicielles.
\end{itemize}
\subsection{Développement Web}
\subsubsection{Découverte du projet principal et études préliminaires}
L'objectif principal de mon stage était la \textbf{digitalisation des processus de travail du département des ressources humaines (RH)}. Avant d'entamer le développement, plusieurs tâches préparatoires ont été réalisées :




\begin{itemize}
    \item Analyse des besoins du département en collaboration avec le responsable RH.
    \item Étude des solutions existantes et identification des axes d'amélioration.
    \item Participation aux réunions pour recueillir les attentes des utilisateurs.
    \item Rédaction d'un cahier des charges et validation des fonctionnalités essentielles avec mon tuteur.
\end{itemize}

\subsubsection{Développement de la solution numérique}
Après cette phase d'étude, j'ai entamé la conception et le développement de l'application. Cela a inclus :
\begin{itemize}
    \item La mise en place de l'environnement de développement (choix des technologies, configuration des outils).
    \item La conception de l'architecture du projet et de la base de données.
    \item Le développement des premières fonctionnalités, notamment \textbf{ l'authentification et la gestion des utilisateurs}.
    \item Des tests et des ajustements basés sur les retours du tuteur et des futurs utilisateurs.

\end{itemize}
Par la suite, d'autres fonctionnalités ont été intégrées, comme :

\begin{itemize}
    \item La gestion des documents administratifs.
    \item L'automatisation de certaines tâches RH.
    \item L'amélioration de l'interface utilisateur pour une meilleure expérience.
\end{itemize}
\subsubsection{Finalisation du projet et restitution}
Dans les dernières semaines du stage, mon travail s'est concentré sur :

\begin{itemize}
    \item La finalisation du développement et la correction des derniers bugs.
    \item La rédaction de la documentation technique et utilisateur.
    \item La présentation du projet aux responsables du département RH et la collecte des feedbacks.
    \item La formation des collaborateurs à l'utilisation de la solution développée.

\end{itemize}
\subsection{Débogage du réseau téléphonique IP de l'entreprise}
Afin d’assurer le bon fonctionnement des communications internes, une intervention sur le réseau téléphonique IP a été menée :
\begin{itemize}
    \item Analyse des dysfonctionnements et identification des causes des interruptions de service.
    \item  Participation aux interventions techniques visant à rétablir la connexion et  la qualité des appels.
    \item Assistance aux utilisateurs pour résoudre les problèmes liés à la téléphonie IP
\end{itemize}

\subsection{Support aux utilisateurs de la plateforme PASSE}
Un accompagnement a été apporté aux utilisateurs de la plateforme PASSE afin d’assurer son bon fonctionnement :
\begin{itemize}
    \item Réponse aux demandes et assistance aux opérateurs économiques rencontrant des difficultés techniques.
\end{itemize}
\subsection{Assistance technique générale}
En complément des missions spécifiques, un support technique global a été assuré au sein de l’organisation :
\begin{itemize}
    \item Dépannage et maintenance des outils informatiques utilisés par le personnel.
    \item Configuration et optimisation des postes de travail selon les besoins des employés.
    \item Sensibilisation aux bonnes pratiques pour garantir la pérennité des équipements et la sécurité des données.
\end{itemize}
\section{Remarques et suggestions}
\section{Bilan et conclusion}
Cette expérience a permis de mieux appréhender les enjeux liés à la gestion d’un service informatique, alliant assistance aux utilisateurs, maintenance technique et développement d’outils numériques.

\clearpage
